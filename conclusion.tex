\chapter{Diskussion}

\section{Probleme}
\subsection{Die Corona-Pandemie}
Die Pandemie hatte ebenfalls auf diese Arbeit einen großen Einfluss. Während der Präsenzkurse konnten die Teilnehmer durch das permanente tragen von Masken schlechter beobachtet werden. So war das herauslesen von Mimik in den Gesichtern erschwert, wie Verwirrung oder Stolz. \\
Dass die Kurse eine gewisse Zeit ausschließlich online stattfinden konnte stellte ebenfalls eine Beeinträchtigung der Beobachtung dar. So waren manche Teilnehmer überhaupt nicht zu erkennen, da die Kamera ausgeschaltet wurde, sobald diese zu dem Programm von WeDo 2.0 wechselten. Es war schwer möglich, den Kindern bei Baufehlern und Programmierfehlern zu helfen, da viele Kinder auch unterschiedliche Geräte zur Videokonferenz und zum Programmieren benutzten, oder nicht in der Lage waren den Bildschirminhalt zu übertragen.\\
Ein besonderes Problem stellte es dar, wenn technische Probleme auftraten, bei denen die Eltern nicht helfen konnten. So traten Probleme mit der Bluetooth Verbindung der Computer mit dem WeDo 2.0 Baustein auf.

\subsection{Der \acrlong{bctt}} \label{problemCTT}
Unsere Teilnehmer, wie auch die Teilnehmer der \acrlong{guc} schnitten bei der ersten Durchführung des \acrshort{bctt} außerordentlich gut ab. Dies könnte daran liegen, dass die Teilnehmer ohnehin bereits ein Interesse für programmierbare Spielzeuge, wie \gls{Lego}-Boost hatten. Es gab jedoch auch einen Altersunterschied. So schnitten die beiden Viertklässler besser ab als die Zweitklässler. Sehr viel deutlicher wurde dieser Unterschied bei dem abschließenden \acrshort{cctt}. Während der Test für die Zweitklässler nahezu unlösbar war, hatten die Viertklässler hier passable Ergebnisse.\\
Ebenfalls wäre es problematisch gewesen, wenn die Kinder zu Beginn und zum Schluss den gleichen Test gemacht hätten, da die beiden Durchführungen lediglich 6 Monate auseinander gelegen hätten.

 

\section{Verbesserungen}
\subsection{Gruppengröße}\label{size}
Um einen Zusammenhang zwischen den \acrlong{mbti} und dem Fortschritt der Kinder zu finden wäre eine größere Stichprobe von Vorteil. Von den Typen hatten wir, wenn diese überhaupt vertreten waren, nur eine sehr kleine Stichprobe. Jedoch stellt dies die Herausforderung, Beobachtungsbögen für jeden Teilnehmer auszufüllen, was bei 8 Kindern bereits alles andere als einfach war.

\subsection{Beobachter}
Wie unter \ref{size} erwähnt, war es bereits nicht einfach, auf die verschiedenen Aspekte der Beobachtungsbögen bei 8 Teilnehmern zu achten, sodass möglicherweise wichtige Beobachtungen nicht erfasst wurden. Ein zusätzlicher Beobachter oder einer der viel Erfahrung mit dem dokumentieren von Lernverhalten und Erfahrungen hat, würde zu deutlich besseren Ergebnissen führen.

\subsection{Der \acrlong{cctt}}
Um das Problem aus \ref{problemCTT} zu lösen, wäre in zukünftigen Arbeiten eine homogene Teilnehmergruppe von ausschließlich Zweit- oder Viertklässlern besser.\\
So wäre der \acrshort{bctt} für die Zweitklässler geeignet, während der \acrshort{cctt} für die Viertklässler verwendet werden könnte.
Ebenfalls sollte die Stichprobe ausgeweitet werden. Dies könnte dadurch geschehen, dass auch andere Teams der \acrshort{fll} den \acrshort{bctt} machen. Dabei sollten verschiedene Gruppen gebildet werden. Die Erste absolviert den Test zu Beginn der Saison, wobei auch gefragt wird, ob ein Teilnehmer zum ersten Mal dabei ist. Die Zweite absolviert den Test am Ende der Saison, wobei auch hier wichtig wäre zu erfahren, wer das erste Mal an der \acrshort{fll} teilgenommen hat. Aus diesen Daten ließe sich herauslesen, welchen Fortschritt diejenigen, die das erste mal teilgenommen haben, hatten. Dabei würde ebenfalls kein Teilnehmer den Test doppelt machen.



