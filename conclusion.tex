\chapter{Diskussion}
Im folgenden Abschnitt wird über die Ergebnisse ein Fazit gezogen, sowie Probleme, welche während der Studie auftraten, berichtet und durch mögliche Verbesserungen ergänzt.
\section{Fazit}

Insgesamt soll die Arbeit zwei Fragen beantworten. Die erste Frage, die geklärt werden muss, ist die Frage nach dem Weg, Kinder mithilfe \gls{Lego} für die Informatik zu begeistern. Generell lässt sich sagen, dass der Weg, der für diese Studie eingeschlagen wurde, erfolgreich war, um Kinder für Informatik zu begeistern. Die Kinder sind zwar noch davon entfernt, sich mit der "richtigen" Informatik zu beschäftigen. Dies liegt vor allem am Alter der Kinder. Jedoch wurde der richtige Grundstein gelegt, um die Kinder zu begeistern, da die Kinder unmittelbar nach dem Wettbewerb und dem damit einhergehenden Ende der Studienarbeit beziehungsweise dem Ende der Kurse an der Dualen Hochschule unbedingt die Kurse weiterführen wollten. Dabei wollten sie nicht nur mit den Sets arbeiten, die im Rahmen der Studie verwendet wurden, sondern auch mit den Lego-Sets für die nächste Alterstufe. Daher kann gesagt werden, dass das Ziel, Kinder für Informatik zu begeistern, über diesen Weg sehr gut möglich ist.\\
Im Vergleich aber mit der Studie in Ägypten zeigt sich, dass die Kurse in Ägypten zumindest in den Beobachtungen bessere Ergebnisse erzielten. Ein logischer Rückschluss daraus wäre, die Themen, die in dieser Studie behandelt wurden, auch hier zu adaptieren, um so eventuell auch bessere Ergebnisse zu erzielen.\\


Die nächste Frage, die es zu beantworten gilt, ist, ob sich Persönlichkeitstypen, welche in den Grundlagen definiert wurden, auf das Lernverhalten der Kinder auswirkt. Dies ist durchaus der Fall. In der Studie wurden verschiedene Persönlichkeitstypen analysiert und das Ergebnis zeigt, dass diese vor allem auf das Lernverhalten starke Auswirkungen haben. Dies ist sichtbar in den unterschiedlichen Entwicklungen, die in Kapitel \ref{sec:personalityAndDevelopment} verglichen wurden. So sind besonders die Border Collies in der Studie ein sehr lernstarker Persönlichkeitstyp, was sich in den Konstanten und häufigen positiven Entwicklungen zeigt. Komplettes Gegenteil zu den Border Collies ist die Gruppe der Pandas, da diese nur wenig Entwicklung zeigte und auch generell eher schlechtere Werte erzielte. In den Computational Thinking Tests zeigte sich ebenfalls ein Unterschied in den Ergebnissen. So sind hier die Elefanten sehr stark gewesen, vor den Erdmännchen und den Border Collies. Schlusslicht war meist der Panda. Für alle Tiere gilt aber, dass die Stichprobe, die in dieser Studie gewählt wurde, zu gering ist, um eine konkrete Aussage zu treffen. Dies wird aber in Kapitel \ref{sec:problems} genauer erläutert. Zudem ist die Aussagekraft des \acrlong{mbti} aufgrund vieler Gründe umstritten. Daher kann nicht mit großer Sicherheit festgestellt werden, ob diese Lernunterschiede tatsächlich den Persönlichkeitstypen der Kinder verschuldet sind oder auf andere Einflüsse wie familiäre Situation, Herkunft der Kinder oder auch Geschlecht zurückzuführen sind.


\section{Probleme}\label{sec:problems}
\subsection*{Die Corona-Pandemie}
Die Pandemie hatte ebenfalls auf diese Arbeit einen großen Einfluss. Während der Präsenzkurse konnten die Teilnehmer durch das permanente tragen von Masken schlechter beobachtet werden. So war das Herauslesen von Mimik in den Gesichtern erschwert, wie beispielsweise Verwirrung oder Stolz. \\
Dass die Kurse eine gewisse Zeit ausschließlich online stattfinden konnte stellte ebenfalls eine Beeinträchtigung der Beobachtung dar. So waren manche Teilnehmer überhaupt nicht zu erkennen, da die Kamera ausgeschaltet wurde, sobald diese zu dem Programm von WeDo 2.0 wechselten. Es war schwer möglich, den Kindern bei Baufehlern und Programmierfehlern zu helfen, da viele Kinder auch unterschiedliche Geräte zur Videokonferenz und zum Programmieren benutzten, oder nicht in der Lage waren den Bildschirminhalt zu übertragen.\\
Ein besonderes Problem stellte es dar, wenn technische Probleme auftraten, bei denen die Eltern nicht helfen konnten. So traten unter  Probleme mit der Bluetooth Verbindung der Computer mit dem WeDo 2.0 Baustein auf.

\subsection*{Der \acrlong{bctt}} \label{sec:problemCTT}
Unsere Teilnehmer schnitten bei der ersten Durchführung des \acrshort{bctt} außerordentlich gut ab. Dies könnte daran liegen, dass die Teilnehmer ohnehin bereits ein Interesse für programmierbare Spielzeuge, wie \gls{Lego}-Boost hatten. Es gab jedoch auch einen Altersunterschied. So schnitten die beiden Viertklässler besser ab als die Zweitklässler. Sehr viel deutlicher wurde dieser Unterschied bei dem abschließenden \acrshort{cctt}. Während der Test für die Zweitklässler nahezu unlösbar war, hatten die Viertklässler hier passable Ergebnisse.\\
Ebenfalls wäre es problematisch gewesen, wenn die Kinder zu Beginn und zum Schluss den gleichen Test gemacht hätten, da die beiden Durchführungen lediglich 6 Monate auseinander gelegen hätten.


\section{Verbesserungen}
\subsection*{Gruppengröße}\label{sec:size}
Um einen Zusammenhang zwischen den \acrlong{mbti} und dem Fortschritt der Kinder zu finden wäre eine größere Stichprobe von Vorteil. Von den Typen, wenn diese überhaupt vertreten waren, gab es  nur eine sehr kleine Stichprobe. Dies war jedoch schon eine Herausforderung für eine Person, alle acht Kinder zu beobachten und deren Verhalten zu dokumentieren.

\subsection*{Beobachter}
Wie unter \ref{sec:size} erwähnt, war es bereits nicht einfach, auf die verschiedenen Aspekte der Beobachtungsbögen bei 8 Teilnehmern zu achten, sodass möglicherweise wichtige Beobachtungen nicht erfasst wurden. Ein zusätzlicher Beobachter oder einer der viel Erfahrung mit dem dokumentieren von Lernverhalten und Erfahrungen hat, würde zu deutlich besseren Ergebnissen führen.

\subsection*{Der \acrlong{cctt}}
Um das Problem aus \ref{sec:problemCTT} zu lösen, wäre in zukünftigen Arbeiten eine homogene Teilnehmergruppe von ausschließlich Zweit- oder Viertklässlern besser.\\
So wäre der \acrshort{bctt} für die Zweitklässler geeignet, während der \acrshort{cctt} für die Viertklässler verwendet werden könnte.
Ebenfalls sollte die Stichprobe ausgeweitet werden. Dies könnte dadurch geschehen, dass auch andere Teams der \acrshort{fll} den \acrshort{bctt} machen. Dabei sollten verschiedene Gruppen gebildet werden. Die Erste absolviert den Test zu Beginn der Saison, wobei auch gefragt wird, ob ein Teilnehmer zum ersten Mal dabei ist. Die Zweite absolviert den Test am Ende der Saison, wobei auch hier wichtig wäre zu erfahren, wer das erste Mal an der \acrshort{fll} teilgenommen hat. Aus diesen Daten ließe sich herauslesen, welchen Fortschritt diejenigen, die das erste Mal teilgenommen haben, hatten. Dabei würde ebenfalls kein Teilnehmer den Test doppelt machen.

\subsection*{Ein deutscher Persönlichkeitstest}
Wie in \ref{Teilnehmer} erwähnt, handelt es sich bei dem Persönlichkeitstest von \citetitle{knowAndLove} um einen Englischsprachigen Test. Dies führt dazu, dass die Interpretation der Fragen und Antworten in der Verantwortung der Eltern, die diesen Test mit ihren Kindern durchgeführt haben, lag. Somit können die Eltern einen Einfluss auf das Ergebnis ihrer Kinder genommen haben. Um den Einfluss Dritter bestmöglich zu Beheben wäre eine einzige Übersetzung des Testes in das Deutsche besser.


