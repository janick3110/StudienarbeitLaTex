\setcounter{chapter}{-1}
\chapter{Abstracts}
\section*{Zusammenfassung}

Diese Studienarbeit siedelt sich im interdisziplinären Fachgebiet Human Computer Interaction als Teilgebiet der Informatik an, in einem der Grand Challenges zu den Themen Lernen mit Technologie und Serious Games. Das Zusammenspiel von Technik, Game, Tutoren und Lernende soll analysiert werde unter Einbezug der verschiedenen Persönlichkeiten in der Gruppe. Als qualitative Studie basiert die Analyse auf der Beschreibung des Lernprozesses während der Lösungsfindung von der Kreativität bis hin zur Durchführung und Evaluation seitens des Kinderteams und aus Sicht der beiden Tutoren.\\
Am Ende dieser Arbeit sollen zwei Fragen beantwortet werden. Die erste Frage ist ob sich Kinder mithilfe der Zuhilfenahme von LEGO und der Teilnahme an einem zugehörigen Wettbewerb für Informatik begeistern lassen. Zudem soll beantwortet werden, ob sich Persönlichkeitscharaktere auf das Lernverhalten auswirken.\\
Dazu werden regelmäßig Kurse mit LEGO WeDO 2.0 durchgeführt, die von einer weiteren Person beobachtet werden. Dabei wird die Mitarbeit, welche unter anderem aus Motivation, Kreativität und Teamarbeit besteht, der Kinder dokumentiert. Die Kinder, die an dieser Studie teilnehmen, führten zuvor Tests durch, welche ihre Persönlichkeiten ermitteln sollte. Zudem wurde zu Beginn und am Ende der Studie ein Test durchgeführt, welcher die Kinder auf ihre Fähigkeit, algorithmisch zu denken, untersuchte.\\
Die Ergebnisse werden mit einer parallel veranstalteten Studie an der \acrlong{guc} verglichen.\\
Die Ergebnisse zeigen, dass Persönlichkeitstypen durchaus eine Rolle für die Lernfähigkeit mit Technologie spielen. Jedoch ist nicht sicher, ob dies wirklich auf die Typen zurückzuführen ist, da es für den Test viel wissenschaftliche Kritik gibt.\\

\section*{Abstract}

This research paper deals with the interdisciplinary subject of Human Computer Interaction. Its aim is to analyse the interplay of technology, games, tutors and students. Additionaly, personality types are also accounted for. The analysis is based on evaluation which will be done by the authors of this paper.\\
At the end of this study, two question will be answered. First, can children be fascinated with computer science if they take part in a competition and a course which uses LEGO kits? Second, are there differences in learning based on their personality types?\\
To analyse this, there were courses done by the authors, which observed the children and documented aspects like motivation, creativity and team work. The children which took part in this solved tests, which documented how well their computational thinking is distinct. This test was done at the beginning and also at the end to see if there is progress.\\
There was also a parallel study at the German University of Cairo, which documented also children. These results were compared with this study to see if there are differences.\\
The final results show that there is definetly a difference in learning through the different groups of personality types. But it is not clear wether the differences are due to the different personality types or not because there is much scientific criticism for this test.