\chapter{Beobachtungen}
Im folgenden Abschnitt werden die subjektiven Beobachtungen der Autoren dokumentiert. Dabei werden die Beobachtungen nach Persönlichkeitstyp kategorisiert.
\section{Präsenzveransatltungen}
\subsection{Border Collies}
Während den Workshops konnte bei allen Probanden, welche aus ihrem Persönlichkeitstest das Ergebnis \textit{Border Collie} erhielten, festgestellt werden, dass ihre Fähigkeit, im Team zu arbeiten, vorhanden war. Bei den beiden Kindern, die diesen Persönlichkeitstyp zugeordnet hatten, hat sich die Teamfähigkeit weder verschlechtert oder verbessert. Für die Kinder war das Arbeiten des Teams wichtig, sie arbeiteten miteinander und schätzen die gegenseitigen Beiträge, so dass keiner im Alleingang arbeitete. Ihr Verhalten gegenüber anderen Teampartnern war bei beiden Kindern in der Regel gleich, nur zu Beginn der Workshops zeigte ein Kind einen leicht arroganten Charakter gegenüber anderen Kindern, welcher sich durch Aussagen wie: \textit{"Mach du das mal, ich kann das schon"}, äußerte. Wenn es darum ging, die Führungsrolle zu übernehmen, konnten beide Teilnehmer gut andere Teampartner anweisen, ohne aber dabei zu dominant zu werden. Beide Kinder kommunizierten in der Regel sehr gut mit anderen Kindern, eines der Kinder hatte jedoch manchmal Schwierigkeiten, sich richtig auszudrücken. Dieses Verhalten ist vermutlich nicht auf die Persönlichkeit, sondern auf einen Sprachfehler zurückzuführen. Dasselbe Kind war jedoch trotz der Schwierigkeiten sich auszudrücken sehr kommunikativ was das Feedback betraf. Die Autoren wurden häufig gefragt, für wie gut sie die Bauwerke des Kindes befanden.


Die Kinder des Persönlichkeitstyps \textit{Border Collie} hatten jedoch Schwierigkeiten, eigene Ideen während der Besprechungsrunden zu bilden. Häufig wurden entweder Ideen von anderen Teilnehmern übernommen oder nur nach Hilfestellung der Autoren, welche das Maß, welches bei der Altersklasse angemessen wäre, übertraf, so dass häufig bei ganz niedrigen Grundlagen gestartet werden musste. Jedoch konnte manchmal auch beobachtet werden, dass die Kinder verschiedene Konzepte ausprobierten, dies allerdings geschah nur sehr selten. Die Kinder taten sich auch häufig schwer, die Dinge auch aus anderen Perspektiven zu betrachten.\\
Ein Abstraktionsverhalten der Kinder wurde kaum festgestellt, nur ganz selten konnte beobachtet werden, dass die Kinder über das Thema hinaus sehen konnten. Ihr Verhalten in den Aspekten Beharrlichkeit und Integrationsfähigkeit wurde von den Autoren als gut dokumentiert. Dies bedeutet, dass die Kinder zwar in der Lage waren, diese Fähigkeiten zu zeigen, dies jedoch nicht sehr häufig geschah oder die Ausprägung der Fähigkeiten nicht stärker war.\\
Generell konnte beobachtet werden, dass die Kinder gegenüber neuen Themen sehr offen eingestellt waren und sich dann auch mit den Themen gedanklich befassten. Ab und zu konnte festgestellt werden, dass die Kinder ein wenig Fantasie zeigten und ab und zu Züge von Humor aufzeigten. Die Kinder zeigten in der Regel sich als sehr spielerische Kinder, was sich dadurch zeigte, dass die Kinder öfters auch mit ihren Bauwerken und anderen Teilen spielten, statt diese nur zu reinen Programmierzwecken zu verwenden.\\
Die Kinder zeigten während der Workshops, besonders beim Implementieren ihrer Programme, dass sie in der Lage sind, logisch zu denken. Die logischen Gedankengänge konnten auch, wenn die Autoren die Kinder dazu befragten, erklärt werden.


Wenn es darum ging, Probleme zu beschreiben, waren beide Kinder dieses Persönlichkeitstyps in der Lage, dies zu tun. Dazu wurde keine Hilfestellung seitens der Autoren benötigt. Erfolgskriterien konnten von den Kindern nicht  selbständig genannt werden, genauso wenig konnten sie das Problem nicht in kleinere Schritte zerlegen, außer die Autoren führten sie mit Fragen durch das Problem. Hin und wieder konnte bei einzelnen Fällen erkannt werden, dass die Kinder in der Lage sind, Muster zu erkennen. Gelernte Konzepte konnten im Regelfall von den Kindern selbständig angewandt werden, ohne dass Hilfe benötigt wurde.\\
Das algorithmische Denken wurde bei den von Lego gestellten Projekten übernommen, da die Software den benötigten Code vorgab und die Kinder nur den Code nachbauen mussten. In den freien Aufgaben, die unabhängig von den genannten Projekten waren, konnten die Kinder in der Regel nachdem die Autoren ihnen den Weg gezeigt hatten, ein algorithmisches Denken an den Tag legen. Die algorithmischen Konzepte, die unter anderem in Abbildung \ref{pdf:observation_sheets} aufgeführt wurden, konnte nach einigen Wochen von den Kindern selbständig angewandt werden, ohne dass externe Hilfe benötigt wurde.\\
Durch die Vorgabe ist auch die Beurteilung der Funktionalität eines Programms hinfällig, weshalb lediglich freie Aufgaben bewertet wurden. Diese Aufgaben wurden in den meisten Fällen selbständig bearbeitet und funktionierten ohne Eingreifen der Autoren. Fehler wurden selbständig behoben, alternative Lösungen wurden jedoch zu selten gefunden, als dass dies hätte bewertet werden können.

\subsection{Elefanten}
\subsection{Erdmännchen}
Die Kinder des Persönlichkeitstyps Erdmännchen konnten zu Beginn sehr gut in einem Team arbeiten, egal ob in Zweier- oder Viererteams. Beide unterscheiden sich jedoch in ihrer Rolle in einem Team. Benny zeigte sehr dominantes Verhalten in seinen Teams, selbst mit anderen Kindern, die ähnlich dominant waren. Henriette dagegen hatte ein eher zurückhaltendes Verhalten im Vergleich zu Benny. Trotz der Zurückhaltung beteiligte sie sich aktiv in den Gruppen, jedoch nicht in der Rolle einer (dominanten) Anführerin, wie es Benny häufig übernahm. Beide waren jedoch in der Lage, gut in ihren Teams zu arbeiten. Zwischen ihren Teampartnern kam es meist immer zu viel guter und sinnvoller Kommunikation. Auch bei Aufgaben, bei denen ihnen absichtlich das Gegenteil ihrer präferierten Rolle zugewiesen wurde, hatten sie keine Probleme, wenn auch es gerade für Benny schwer viel, jemand anderem seine gewohnte Rolle zu überlassen. In ihren Gruppenarbeiten wurden ihre Teampartner von ihnen immer gut unterstützt.

Auch bei den Ideen unterschieden sich beide Kinder. Benny brachte sich während den Besprechungsrunden über die verschiedenen Themen häufig ein, zeigte dabei auch verschiedene Perspektiven auf. Henriette dagegen zeigte eher weniger Flexibilität und war auch häufig abgelenkt. Ideen von beiden Kindern waren oft sehr originell und sehr kreativ, gerade Benny zeigte hin und wieder mit Ideen, welche über dem Niveau, welches in der Altersklasse normal wäre, lagen, dass er über ein hohes Maß an Wissen und Verständnis verfügte. Hier wurden physikalische Konzepte wie Reibung und Anpressdruck erwähnt, welche für dieses Alter nicht selbstverständlich sind.\\
Im Bereich des emergierenden Denkens fehlte Henriette die Fähigkeit, aus konkreten Dingen zu abstrahiern. Dies gelang Benny deutlich besser. Auch war ein deutlicher Unterschied in der Beharrlichkeit beider Kinder zu sehen. Benny arbeitete deutlich beharrlicher an den einzelnen Projekten, deutlich detaillierter als es bei Henriette der Fall war. Bei ihr trat das Problem auf, dass sie sich oft zu sehr zu leicht ablenken ließ. Während Benny auch meist ohne größere Mühe zwischen Dingen Verbindungen knüpfen konnte, waren dieselben Verbindungen für Henriette öfters nicht verständlich. \\
Bei Benny zeigte sich schon zu Beginn ein großes Interesse in die Themen und war immer sehr aufgeschlossen gegenüber neuen Themen und Konzepten. Er versuchte Unbekanntes zu verstehen und nicht nur wie es anzuwenden war. Henriette versuchte sich ebenfalls in neue Thematiken einzuarbeiten, jedoch nicht so aktiv wie es Benny versuchte. Benny, der von Anfang an sehr aktiv und redefreudig war, zeigte hin und wieder fantastische und humoristische Züge. Im Gegensatz zu ihm war Henriette deutlich schüchterner als er, welches sich aber im Laufe der Termine immer besserte. Sie verhielt sich auch deutlich kindlicher und verspielter, als es Benny tat. Auch sie zeigte, dass sie durchaus in der Lage war, mit Fantasie umzugehen. Dies wurde beispielsweise dadurch ersichtlich, dass sie in Bauwerke, welche nur aus wenigen Elementen bestanden, viele Dinge hineininterpretieren konnte.\\
Während den einzelnen Aufgaben zeigte sich, dass Benny oftmals besser in der Lage war, logische Schritte zu durchlaufen, während bei Henriette oftmals Verbindungen und Struktur fehlten.

Beide Kinder waren in der Lage, Probleme zu beschreiben, sobald sie darauf angesprochen wurden, ohne dass sie Hilfe benötigten. Es war möglich, sie nach den Kriterien zu fragen, die benötigt wurden, um die Probleme zu lösen. Lediglich die Zerlegung gelang Benny besser als Henriette.\\
Benny konnte Muster selbständig erkennen, während Henriette zwar auch in der Lage dazu war, jedoch dazu eine Hilfestellung seitens der Autoren benötigte. Nach anfänglichen Einführungen fiel es für beide Kinder leicht, bereits gelernte Konzepte in neuen Projekte umzusetzen.\\
Im Bereich algorithmischen Denken tat sich Benny sehr leicht, die Sequenzen von Schritten zu bilden und einzelne Schritte zu implementieren, während bei Henriette oft die Autoren eine Hilfestellung geben mussten. Die Konzepte wie If-Bedingungen und Ähnliches konnten sie selbständig in die Aufgaben miteinfließen lassen, nachdem sie diese erklärt bekommen haben.\\
Wie bereits erwähnt konnten die geführten Lego-Projekte selbständig gelöst werden. Bei den eigenen Projekten konnten die Programme selbständig funktionsfähig gemacht werden, auch auftretende Probleme wurden von den Kindern ohne weitere Hilfestellungen gelöst. Auf die Frage, warum dadurch das Ausgangsproblem gelöst wurde, konnten sie in der Regel ohne Führungshilfen oder andere Hilfestellungen der Autoren beantworten.\\
Während Benny selbständig die wichtigsten Details der Lösung aufzeigen konnte, war Henriette auf die Unterstützung durch die Autoren angewiesen, konnte aber dadurch ebenfalls die Details nennen.\\

Gegen Ende der Termine verschlechterte sich jedoch bei beiden die Leistung. Ihre Gruppen, die bisher immer sehr stark waren und gute Ergebnisse lieferten, hatten große Probleme, fertig zu werden und sinnvolle Aufgaben zu erledigen. Die Kommunikation untereinander verstummte, stattdessen wurde sehr viel mit anderen Gruppen geredet über Thematiken, die nichts mit den Workshops zu tun hatten. Ein Einschreiten der Autoren reduzierte zwar für kurze Zeit diesen Austausch, jedoch nach einiger Zeit nahm dieser Effekt wieder ab. Zusätzlich zu diesen Gesprächen wurden beide Kinder lauter, riefen viel durch den Raum und rannten durch den Hörsaal, in dem der Workshop stattfand. Dadurch störten sie einerseits ihren eigenen Fortschritt als auch den anderer Gruppen. Die Gruppen der beiden Kinder wurden deshalb am Ende oft nur knapp, teilweise sogar gar nicht fertig mit Aufgaben, deren Zeitaufwand deutlich weniger Zeit erforderte, als sie für die Aufgabe bekommen hatten.\\
Benny wurde gegen Ende gemeiner, versuchte Bauwerke anderer Gruppen zu zerstören, die zwar nicht für die eigentliche Aufgabe wichtig war, diese Gruppe in dieses Objekt jedoch sehr viel Fantasie steckte. Daher war hier ein Einschreiten der Autoren deutlich häufiger erforderlich.

\subsection{Panda}
Die folgenden Beobachtungen sind nur für eine Person. Daher kann es vorkommen, dass Personen, welcher dem gleichen Persönlichkeitstyp angehören, andere Verhaltensmuster aufweisen.

Das Kind, welches diesem Persönlichkeitstyp zugeordnet wurde, tat sich sehr schwer, was die Arbeit im Team anbelangt. Generell war zu beobachten, dass dem Kind die Arbeit im Team wichtig war, jedoch auch nicht allzu wichtig. Was das Lernen im Team betrifft, ist es sehr schwer, hierzu eine Beobachtung durchzuführen. Sara zeigte im Wesentlichen keine größeren Fortschritte in den Bereichen, die in den Workshops den Kindern vermittelt wurden. Einerseits arbeitete sie in einer Gruppe, wurde aber von den anderen Teilnehmern immer stark dominiert, so dass sie sich sehr zurückgezogen hatte. Die Autoren mussten hierbei sie immer wieder auffordern, an den Projekten auch geistig teilzunehmen und nicht nur zuzusehen. Durch die Zurückgezogenheit leidet auch ihre Fähigkeit zu kommunizieren, da kaum Kommunikation stattfindet und sie nicht nach Feedback fragt beziehungsweise gibt. Aufgrund ihrer rezessiven Verhaltensweise war Sara nicht in der Lage, in einer Gruppe als Führungskraft aufzutreten und andere Gruppenmitglieder zu delegieren. Trotz ihrem Verhalten konnten gegen Ende Verbesserungen im Teamverhalten beobachtet werden, was durch eine regere Mitarbeit in den Projekten sichtbar wurde.

Sara zeigte während der Workshops nur selten einen Drang zu Neuem. Zwar versuchte sie, hin und wieder dabei zu sein, hatte aber sonst damit oft Probleme. Für ihre emotionale Sensitivität kann nur eine Aussage über ihre Schüchternheit getroffen werden. Dadurch zeigte Sara während den Workshops nur selten bis gar nicht, dass sie die Fähigkeit für Humor besitzt. Bei den Bauwerken und Ideen konnte sie ab und zu ein wenig Fantasie zeigen, aber ansonsten blieb diese Eigenschaft im Verborgenen.\\
Das logische Denken konnte durch ihre geringe Mitarbeit nur bei Aufforderung beobachtet werden, jedoch fiel ihr logische Schlüsse zu ziehen schwer.

Generell kann über ihre Computational Thinking Skills gesagt werden, dass sie für die meisten Attribute zwar in der Lage war, jedoch nur nach Aufforderung und zeigen durch die Autoren. Mithilfe von Hilfestellungen und dem Führen durch Probleme und Themen war sie dann doch in der Lage ein Problem zu zerlegen und zu erkennen, was nötig ist, um dieses Problem zu lösen. Dasselbe galt auch für bereits gelernte Konzepte; Sara musste oft noch einmal an die Konzepte, die in den letzten Stunden durchgearbeitet wurden, herangeführt werden.\\
Wie bei den anderen Gruppen bereits erwähnt, wurden für das algorithmische Denken nur die freien Aufgaben abseits der geführten Kurse beobachtet. Hier hatte Sara jedoch einige Probleme, stand auch hier wieder im Schatten ihrer Teampartner, welche das meiste für sie erledigten. Zwar haben die Programme ihres Teams funktioniert, ihre Mitarbeit an diesen ist jedoch nie sehr hoch gewesen, nur nach mehrmaligem Auffordern. Alternative Lösungen wurden nicht geboten.\\
Ihre Fähigkeiten der Abstraktion waren auch nur durch sehr viele Hilfestellungen möglich, ein Hinterfragen der Lösung warum diese jetzt funktioniert und weshalb ein Teil der Implementierung wichtig ist für die Lösung konnte sie nur sehr schwer durchführen.

\section{Onlineveranstaltungen}