\chapter{Beobachtungen}
Im folgenden Abschnitt werden die subjektiven Beobachtungen der Autoren dokumentiert. Dabei werden die Beobachtungen nach Persönlichkeitstyp kategorisiert.
\section{Präsenzveranstaltungen}
\subsection*{Border Collies}
Während den Workshops konnte bei allen Probanden, welche aus ihrem Persönlichkeitstest das Ergebnis \textit{Border Collie} erhielten, festgestellt werden, dass ihre Fähigkeit, im Team zu arbeiten, vorhanden war. Bei den beiden Kindern, die diesen Persönlichkeitstyp zugeordnet hatten, hat sich die Teamfähigkeit weder verschlechtert oder verbessert. Für die Kinder war das Arbeiten des Teams wichtig, sie arbeiteten miteinander und schätzen die gegenseitigen Beiträge, so dass keiner im Alleingang arbeitete. Ihr Verhalten gegenüber anderen Teampartnern war bei beiden Kindern in der Regel gleich, nur zu Beginn der Workshops zeigte ein Kind einen leicht arroganten Charakter gegenüber anderen Kindern, welcher sich durch Aussagen wie: \textit{"Mach du das mal, ich kann das schon"}, äußerte. Wenn es darum ging, die Führungsrolle zu übernehmen, konnten beide Teilnehmer gut andere Teampartner anweisen, ohne aber dabei zu dominant zu werden. Beide Kinder kommunizierten in der Regel sehr gut mit anderen Kindern, eines der Kinder hatte jedoch manchmal Schwierigkeiten, sich richtig auszudrücken. Dieses Verhalten ist vermutlich nicht auf die Persönlichkeit, sondern auf einen Sprachfehler zurückzuführen. Dasselbe Kind war jedoch trotz der Schwierigkeiten sich auszudrücken sehr kommunikativ was das Feedback betraf. Die Autoren wurden häufig gefragt, für wie gut sie die Bauwerke des Kindes befanden.


Die Kinder des Persönlichkeitstyps \textit{Border Collie} hatten jedoch Schwierigkeiten, eigene Ideen während der Besprechungsrunden zu bilden. Häufig wurden entweder Ideen von anderen Teilnehmern übernommen oder nur nach Hilfestellung der Autoren, welche das Maß, welches bei der Altersklasse angemessen wäre, übertraf, so dass häufig bei ganz niedrigen Grundlagen gestartet werden musste. Jedoch konnte manchmal auch beobachtet werden, dass die Kinder verschiedene Konzepte ausprobierten, dies allerdings geschah nur sehr selten. Die Kinder taten sich auch häufig schwer, die Dinge auch aus anderen Perspektiven zu betrachten.\\
Ein Abstraktionsverhalten der Kinder wurde kaum festgestellt, nur ganz selten konnte beobachtet werden, dass die Kinder über das Thema hinaus sehen konnten. Ihr Verhalten in den Aspekten Beharrlichkeit und Integrationsfähigkeit wurde von den Autoren als gut dokumentiert. Dies bedeutet, dass die Kinder zwar in der Lage waren, diese Fähigkeiten zu zeigen, dies jedoch nicht sehr häufig geschah oder die Ausprägung der Fähigkeiten nicht stärker war.\\
Generell konnte beobachtet werden, dass die Kinder gegenüber neuen Themen sehr offen eingestellt waren und sich dann auch mit den Themen gedanklich befassten. Ab und zu konnte festgestellt werden, dass die Kinder ein wenig Fantasie zeigten und ab und zu Züge von Humor aufzeigten. Die Kinder zeigten in der Regel sich als sehr spielerische Kinder, was sich dadurch zeigte, dass die Kinder öfters auch mit ihren Bauwerken und anderen Teilen spielten, statt diese nur zu reinen Programmierzwecken zu verwenden.\\
Die Kinder zeigten während der Workshops, besonders beim Implementieren ihrer Programme, dass sie in der Lage sind, logisch zu denken. Die logischen Gedankengänge konnten auch, wenn die Autoren die Kinder dazu befragten, erklärt werden.


Wenn es darum ging, Probleme zu beschreiben, waren beide Kinder dieses Persönlichkeitstyps in der Lage, dies zu tun. Dazu wurde keine Hilfestellung seitens der Autoren benötigt. Erfolgskriterien konnten von den Kindern nicht  selbständig genannt werden, genauso wenig konnten sie das Problem nicht in kleinere Schritte zerlegen, außer die Autoren führten sie mit Fragen durch das Problem. Hin und wieder konnte bei einzelnen Fällen erkannt werden, dass die Kinder in der Lage sind, Muster zu erkennen. Gelernte Konzepte konnten im Regelfall von den Kindern selbständig angewandt werden, ohne dass Hilfe benötigt wurde.\\
Das algorithmische Denken wurde bei den von Lego gestellten Projekten übernommen, da die Software den benötigten Code vorgab und die Kinder nur den Code nachbauen mussten. In den freien Aufgaben, die unabhängig von den genannten Projekten waren, konnten die Kinder in der Regel nachdem die Autoren ihnen den Weg gezeigt hatten, ein algorithmisches Denken an den Tag legen. Die algorithmischen Konzepte, die unter anderem in Abbildung \ref{pdf:observation_sheets} aufgeführt wurden, konnte nach einigen Wochen von den Kindern selbständig angewandt werden, ohne dass externe Hilfe benötigt wurde.\\
Durch die Vorgabe ist auch die Beurteilung der Funktionalität eines Programms hinfällig, weshalb lediglich freie Aufgaben bewertet wurden. Diese Aufgaben wurden in den meisten Fällen selbständig bearbeitet und funktionierten ohne Eingreifen der Autoren. Fehler wurden selbständig behoben, alternative Lösungen wurden jedoch zu selten gefunden, als dass dies hätte bewertet werden können.

\subsection*{Elefanten}
Bei zwei der drei Kinder, bei denen der Persönlichkeitstest als Ergebnis Elefant ergab, konnte festgestellt werden, dass die Kinder in der Lage waren, gemeinsam im Team zu arbeiten. Heinz zeigte jedoch zu Beginn im Vergleich zu den anderen beiden Kinder ein eher dominierendes Verhalten gegenüber seiner Teampartner, was der Fähigkeit, im Team zu arbeiten, schadete. Während sich das Verhalten im weiteren Verlauf seltener auftrat und er besser in der Lage war, mit anderen zu arbeiten, verschlechterte sich bei den anderen Kindern die Teamarbeit. Alle drei Kinder gaben selten bzw. gar kein Feedback, lediglich Heinz fragte öfters nach Feedback, jedoch nur, um erneut nachzufragen, was er jetzt zu tun hatte. Während Heinz Probleme hatte, seine Teampartner miteinzubeziehen, konnten die beiden anderen Kinder ihre Partner miteinbeziehen, auch wenn sie manchmal etwas mehr zurückhaltender waren im Vergleich zu ihren Partnern. Jedoch verschlechterte sich dies ebenfalls im Verlaufe der Kurse. Heinz lernte im Verlauf, mit Teampartnern zu kommunizieren, was den anderen Kindern schon zu Beginn gelang. Moritz übernahm während den Kursen häufig die Rolle des Anführers, welche er gut ausübte. Lulu teilte sich die Rolle mit ihren Teampartnern, was in der Regel gut funktionierte. Heinz dagegen war eher ein Einzelkämpfer und führte sein Team nur selten an. Während sich das gegen Ende verbesserte, ließ bei den anderen beiden Kinder die Qualität ihrer Führungsrolle deutlich nach.

Alle drei Kinder waren sehr gute Ideengeber während der Besprechungen, ihre Ideen waren in der Regel immer sehr gute Einfälle. Von den drei Kindern war die Originalität von Moritz sehr gut, die anderen beiden Kinder hatten zwar eine gute, aber nicht übermäßig gute Fähigkeit, originelle Ideen zu bilden. Moritz war außerdem sehr flexibel in der Ideenbildung, während die anderen beiden eher Probleme damit hatten.Heinz hatte sehr große Probleme zu abstrahieren. Er hatte sehr große Schwierigkeiten, aus konkreten Dingen sich darunter etwas abstrakteres vorzustellen.\\
Für Moritz war Abstraktion weniger schwierig, er konnte, wenn gefordert, zu abstrahieren. Für Lulu war dies ebenfalls möglich.Das detailreiche Arbeiten war für Heinz eine große Herausforderung, durch die ganzen Termine hinweg fragte er alle fünf Minuten, was er jetzt machen soll. Moritz arbeitete dagegen sehr fokussiert an den Aufgaben und arbeitet diese auch häufig mit einem hohen Detailgrad aus. Gegen Ende wurde dies jedoch weniger. Lulu arbeitete in der Regel mit einem gewissen Detailgrad, welcher zwar nicht sehr hoch war, jedoch trotzdem noch ein gutes Maß an Details aufwies.Die Fähigkeit zu integrieren war unterschiedlich ausgeprägt. Moritz gelang es oft, Verbindungen zwischen einzelnen Dingen zu knüpfen, Lulu konnte dies ebenfalls genauso gut gut. Nur Heinz hatte Schwierigkeiten, auch teilweise einfache Verbindungen zu knüpfen.\\
Zu Beginn der Termine war Moritz sehr offen gegenüber neuen Themen, zeigte oftmals auch ein sehr großes Interesse an weitergehenden Einblicken in die Themen. Zudem kam ein großer Verbesserungsdrang für seine Bauwerke und wollte dafür Wissen von den Autoren. Lulu war ebenfalls sehr offen gegenüber Neuem, wenn auch nicht so aufgeschlossen wie es Moritz war. Bei beiden Kindern zeigte sich gegen Ende jedoch ein Trend, bei dem sich ihre Offenheit immer weiter reduzierte und sie Projekten weniger offen waren. Bei Heinz dagegen war eine konstante Skepsis gegen neue Themen bemerkbar. Er hatte häufig Probleme, sich in die neuen Themen einzuarbeiten. Alle drei Kinder zeigten hier und da Humor. Für zwei der drei Elefanten war es keine große Schwierigkeit, Fantasie zu zeigen, für Heinz jedoch war es teilweise schon sehr schwierig bestehende Konzepte zu bestehen.\\
Bei Moritz war das logische Denken in der Regel gut, bei den anderen beiden Kindern eher durchwachsen.

Moritz konnte meistens die Problematiken der behandelten Themen gut definieren. Wenn man ihm half, wie er die Erfolgskriterien finden kann, konnte er diese nennen. Das Zerlegen der Probleme benötigte am Anfang wenig Hilfe, gegen Ende mussten die Autoren hier und da Unterstützung bieten. Wie Moritz konnte auch Lulu die Probleme in den Workshops definieren. Bei den Erfolgskriterien konnte nichts beobachtet werden, das Zerlegen in kleinere Teile dagegen konnte mit Hilfe durchgeführt werden. Heinz zeigte beim Nennen und Zerlegen eines Problems sowie beim Nennen eines Erfolgskriteriums große Defizite, war aber nach etwas längerer und intensiverer Hilfe seitens der Autoren doch in der Lage, Genanntes durchzuführen.\\
Muster und Konzepte, die aus vorherigen Terminen bekannt waren, wurden von Heinz oftmals nur nach Hilfe verstanden. Lulu benötigte zwar für Muster mehr Hilfe, konnte aber die Konzepte häufig selbständig anwenden. Moritz zeigte bei den Konzepten durch das Hinzufügen eigener Ideen, das er in der Lage war, die gelernten Konzepte zu verstehen und nicht nur anzuwenden.\\
Im Bereich des algorithmischen Denkens zeigte Lulu, dass sie in der Regel dazu fähig war, größere Schritte zu definieren und sie in kleineren Schritte zu zerlegen. Algorithmische Konzepte wie Schleifen waren für sie nach anfänglicher Einlernphase kein Problem. Heinz hatte dagegen mehr Probleme, die großen Schritte zu definieren. Das Implementieren lief in der Regel selbständig ab, auch wenn er Probleme hatte mit Schleifen oder Ähnlichem. Für Moritz hingegen war es nicht sehr schwierig, die Schritte selbständig zu definieren und implementieren, in einigen Fällen ergänzte er die Implementierungen mit eigenen Ideen. Mit den Konzepten des algorithmischen Denkens konnte er ebenfalls meist selbständig umgehen.\\
Da wie bereits erwähnt Lego einen Teil der Programmierung abnahm, konnte für die Evaluation nur eine beschränkte Beobachtung durchgeführt werden. Die freien Programme der Kinder funktionierten meist selbständig, ohne dass Hilfe notwendig wurde. Fehler wurden von den drei Kindern ebenfalls selbständig behoben. Moritz und Lulu waren in der Lage zu erklären, warum ihr Programm jetzt genau das macht, was es sollte, während dies bei Heinz nicht beobachtbar war. Alternative Lösungen waren selten gefordert, wenn sie gefordert wurden war meist Hilfe notwendig.\\
Für den letzten Aspekt des Computational Thinking, der Abstraktion, konnte Moritz die wichtigsten Details sowohl von der Lösung als auch generell nennen. Wenn gefragt, konnte er den Autoren selbständig erklären, wie das Erfolgskriterium mit der Lösung zusammenhängt. Heinz benötigte auch für die Abstraktion sehr viel Hilfe, das Nennen der Details gelang oftmals nur nach Hilfe der Autoren, genauso wie das Knüpfen eines Zusammenhangs zwischen Lösung und Erfolgskriterien.  Lulu konnte meistens die Details selbständig nennen, nur mit den Zusammenhängen hatte sie manchmal Schwierigkeiten, sodass größere Hilfe notwendig war.

\subsection*{Erdmännchen}
Die Kinder des Persönlichkeitstyps Erdmännchen konnten zu Beginn sehr gut in einem Team arbeiten, egal ob in Zweier- oder Viererteams. Beide unterscheiden sich jedoch in ihrer Rolle in einem Team. Benny zeigte sehr dominantes Verhalten in seinen Teams, selbst mit anderen Kindern, die ähnlich dominant waren. Henriette dagegen hatte ein eher zurückhaltendes Verhalten im Vergleich zu Benny. Trotz der Zurückhaltung beteiligte sie sich aktiv in den Gruppen, jedoch nicht in der Rolle einer (dominanten) Anführerin, wie es Benny häufig übernahm. Beide waren jedoch in der Lage, gut in ihren Teams zu arbeiten. Zwischen ihren Teampartnern kam es meist immer zu viel guter und sinnvoller Kommunikation. Auch bei Aufgaben, bei denen ihnen absichtlich das Gegenteil ihrer präferierten Rolle zugewiesen wurde, hatten sie keine Probleme, wenn auch es gerade für Benny schwer viel, jemand anderem seine gewohnte Rolle zu überlassen. In ihren Gruppenarbeiten wurden ihre Teampartner von ihnen immer gut unterstützt.

Auch bei den Ideen unterschieden sich beide Kinder. Benny brachte sich während den Besprechungsrunden über die verschiedenen Themen häufig ein, zeigte dabei auch verschiedene Perspektiven auf. Henriette dagegen zeigte eher weniger Flexibilität und war auch häufig abgelenkt. Ideen von beiden Kindern waren oft sehr originell und sehr kreativ, gerade Benny zeigte hin und wieder mit Ideen, welche über dem Niveau, welches in der Altersklasse normal wäre, lagen, dass er über ein hohes Maß an Wissen und Verständnis verfügte. Hier wurden physikalische Konzepte wie Reibung und Anpressdruck erwähnt, welche für dieses Alter nicht selbstverständlich sind.\\
Im Bereich des emergierenden Denkens fehlte Henriette die Fähigkeit, aus konkreten Dingen zu abstrahiern. Dies gelang Benny deutlich besser. Auch war ein deutlicher Unterschied in der Beharrlichkeit beider Kinder zu sehen. Benny arbeitete deutlich beharrlicher an den einzelnen Projekten, deutlich detaillierter als es bei Henriette der Fall war. Bei ihr trat das Problem auf, dass sie sich oft zu sehr zu leicht ablenken ließ. Während Benny auch meist ohne größere Mühe zwischen Dingen Verbindungen knüpfen konnte, waren dieselben Verbindungen für Henriette öfters nicht verständlich. \\
Bei Benny zeigte sich schon zu Beginn ein großes Interesse in die Themen und war immer sehr aufgeschlossen gegenüber neuen Themen und Konzepten. Er versuchte Unbekanntes zu verstehen und nicht nur wie es anzuwenden war. Henriette versuchte sich ebenfalls in neue Thematiken einzuarbeiten, jedoch nicht so aktiv wie es Benny versuchte. Benny, der von Anfang an sehr aktiv und redefreudig war, zeigte hin und wieder fantastische und humoristische Züge. Im Gegensatz zu ihm war Henriette deutlich schüchterner als er, welches sich aber im Laufe der Termine immer besserte. Sie verhielt sich auch deutlich kindlicher und verspielter, als es Benny tat. Auch sie zeigte, dass sie durchaus in der Lage war, mit Fantasie umzugehen. Dies wurde beispielsweise dadurch ersichtlich, dass sie in Bauwerke, welche nur aus wenigen Elementen bestanden, viele Dinge hineininterpretieren konnte.\\
Während den einzelnen Aufgaben zeigte sich, dass Benny oftmals besser in der Lage war, logische Schritte zu durchlaufen, während bei Henriette oftmals Verbindungen und Struktur fehlten.

Beide Kinder waren in der Lage, Probleme zu beschreiben, sobald sie darauf angesprochen wurden, ohne dass sie Hilfe benötigten. Es war möglich, sie nach den Kriterien zu fragen, die benötigt wurden, um die Probleme zu lösen. Lediglich die Zerlegung gelang Benny besser als Henriette.\\
Benny konnte Muster selbständig erkennen, während Henriette zwar auch in der Lage dazu war, jedoch dazu eine Hilfestellung seitens der Autoren benötigte. Nach anfänglichen Einführungen fiel es für beide Kinder leicht, bereits gelernte Konzepte in neuen Projekte umzusetzen.\\
Im Bereich algorithmischen Denken tat sich Benny sehr leicht, die Sequenzen von Schritten zu bilden und einzelne Schritte zu implementieren, während bei Henriette oft die Autoren eine Hilfestellung geben mussten. Die Konzepte wie If-Bedingungen und Ähnliches konnten sie selbständig in die Aufgaben miteinfließen lassen, nachdem sie diese erklärt bekommen haben.\\
Wie bereits erwähnt konnten die geführten Lego-Projekte selbständig gelöst werden. Bei den eigenen Projekten konnten die Programme selbständig funktionsfähig gemacht werden, auch auftretende Probleme wurden von den Kindern ohne weitere Hilfestellungen gelöst. Auf die Frage, warum dadurch das Ausgangsproblem gelöst wurde, konnten sie in der Regel ohne Führungshilfen oder andere Hilfestellungen der Autoren beantworten.\\
Während Benny selbständig die wichtigsten Details der Lösung aufzeigen konnte, war Henriette auf die Unterstützung durch die Autoren angewiesen, konnte aber dadurch ebenfalls die Details nennen.\\

Gegen Ende der Termine verschlechterte sich jedoch bei beiden die Leistung. Ihre Gruppen, die bisher immer sehr stark waren und gute Ergebnisse lieferten, hatten große Probleme, fertig zu werden und sinnvolle Aufgaben zu erledigen. Die Kommunikation untereinander verstummte, stattdessen wurde sehr viel mit anderen Gruppen geredet über Thematiken, die nichts mit den Workshops zu tun hatten. Ein Einschreiten der Autoren reduzierte zwar für kurze Zeit diesen Austausch, jedoch nach einiger Zeit nahm dieser Effekt wieder ab. Zusätzlich zu diesen Gesprächen wurden beide Kinder lauter, riefen viel durch den Raum und rannten durch den Hörsaal, in dem der Workshop stattfand. Dadurch störten sie einerseits ihren eigenen Fortschritt als auch den anderer Gruppen. Die Gruppen der beiden Kinder wurden deshalb am Ende oft nur knapp, teilweise sogar gar nicht fertig mit Aufgaben, deren Zeitaufwand deutlich weniger Zeit erforderte, als sie für die Aufgabe bekommen hatten.\\
Benny wurde gegen Ende gemeiner, versuchte Bauwerke anderer Gruppen zu zerstören, die zwar nicht für die eigentliche Aufgabe wichtig war, diese Gruppe in dieses Objekt jedoch sehr viel Fantasie steckte. Daher war hier ein Einschreiten der Autoren deutlich häufiger erforderlich.

\subsection{Panda}
Die folgenden Beobachtungen sind nur für eine Person. Daher kann es vorkommen, dass Personen, welcher dem gleichen Persönlichkeitstyp angehören, andere Verhaltensmuster aufweisen.

Das Kind, welches diesem Persönlichkeitstyp zugeordnet wurde, tat sich sehr schwer, was die Arbeit im Team anbelangt. Generell war zu beobachten, dass dem Kind die Arbeit im Team wichtig war, jedoch auch nicht allzu wichtig. Was das Lernen im Team betrifft, ist es sehr schwer, hierzu eine Beobachtung durchzuführen. Sara zeigte im Wesentlichen keine größeren Fortschritte in den Bereichen, die in den Workshops den Kindern vermittelt wurden. Einerseits arbeitete sie in einer Gruppe, wurde aber von den anderen Teilnehmern immer stark dominiert, so dass sie sich sehr zurückgezogen hatte. Die Autoren mussten hierbei sie immer wieder auffordern, an den Projekten auch geistig teilzunehmen und nicht nur zuzusehen. Durch die Zurückgezogenheit leidet auch ihre Fähigkeit zu kommunizieren, da kaum Kommunikation stattfindet und sie nicht nach Feedback fragt beziehungsweise gibt. Aufgrund ihrer rezessiven Verhaltensweise war Sara nicht in der Lage, in einer Gruppe als Führungskraft aufzutreten und andere Gruppenmitglieder zu delegieren. Trotz ihrem Verhalten konnten gegen Ende Verbesserungen im Teamverhalten beobachtet werden, was durch eine regere Mitarbeit in den Projekten sichtbar wurde.

Sara zeigte während der Workshops nur selten einen Drang zu Neuem. Zwar versuchte sie, hin und wieder dabei zu sein, hatte aber sonst damit oft Probleme. Für ihre emotionale Sensitivität kann nur eine Aussage über ihre Schüchternheit getroffen werden. Dadurch zeigte Sara während den Workshops nur selten bis gar nicht, dass sie die Fähigkeit für Humor besitzt. Bei den Bauwerken und Ideen konnte sie ab und zu ein wenig Fantasie zeigen, aber ansonsten blieb diese Eigenschaft im Verborgenen.\\
Das logische Denken konnte durch ihre geringe Mitarbeit nur bei Aufforderung beobachtet werden, jedoch fiel ihr logische Schlüsse zu ziehen schwer.

Generell kann über ihre Computational Thinking Skills gesagt werden, dass sie für die meisten Attribute zwar in der Lage war, jedoch nur nach Aufforderung und zeigen durch die Autoren. Mithilfe von Hilfestellungen und dem Führen durch Probleme und Themen war sie dann doch in der Lage ein Problem zu zerlegen und zu erkennen, was nötig ist, um dieses Problem zu lösen. Dasselbe galt auch für bereits gelernte Konzepte; Sara musste oft noch einmal an die Konzepte, die in den letzten Stunden durchgearbeitet wurden, herangeführt werden.\\
Wie bei den anderen Gruppen bereits erwähnt, wurden für das algorithmische Denken nur die freien Aufgaben abseits der geführten Kurse beobachtet. Hier hatte Sara jedoch einige Probleme, stand auch hier wieder im Schatten ihrer Teampartner, welche das meiste für sie erledigten. Zwar haben die Programme ihres Teams funktioniert, ihre Mitarbeit an diesen ist jedoch nie sehr hoch gewesen, nur nach mehrmaligem Auffordern. Alternative Lösungen wurden nicht geboten.\\
Ihre Fähigkeiten der Abstraktion waren auch nur durch sehr viele Hilfestellungen möglich, ein Hinterfragen der Lösung warum diese jetzt funktioniert und weshalb ein Teil der Implementierung wichtig ist für die Lösung konnte sie nur sehr schwer durchführen.


\subsection{Gruppenkonstellationen}
Während den Terminen arbeiteten die Kinder oft zusammen. Die Teamauswahl erfolgte meist selbständig. Wenn die Autoren die Teams bildeten, wurde darauf geachtet, dass auch dominantere Kinder sich den Anweisungen anderer Kinder fügen müssen, beispielsweise durch konkretes Aufteilen der Aufgaben, dass ein Kind die Anweisungen geben muss und nicht bauen darf und das andere Kind die Anweisungen seines Partners ausführen muss.\\
Wenn die Kinder selbständig ihre Partner wählen durften, bildeten sich meist die folgenden Gruppen:
\begin{itemize}
	\item Benny (Erdmännchen) \& Moritz (Elefant)
	\item Henriette (Erdmännchen) \& Lulu (Elefant)
	\item Jonas (Border Collie) \& Mario (Border Collie)
	\item Sara (Panda) \& Heinz (Elefant)
\end{itemize}
Zur letzteren Gruppe muss angemerkt werden, dass diese beiden oft zusammenarbeiten mussten, weil keine anderen Teamkonstellationen mehr verfügbar waren. Alle anderen Teamkonstellation bestanden jeweils aus befreundeten Kindern.\\
In manchen Fällen war es notwendig, dass zwei Vierergruppen gebildet wurden. Die Gruppen, die sich dann bildeten, sahen wie folgt aus:
\begin{itemize}
	\item Benny (Erdmännchen) \& Moritz (Elefant) \& Henriette (Erdmännchen) \& Lulu (Elefant)
	\item Jonas (Border Collie) \& Mario (Border Collie) \& Sara (Panda) \& Heinz (Elefant)
\end{itemize}
Die Konstellationen basierten auf den Schulen, die die Kinder besuchten. Eine Gruppe bestand aus Kindern von genau einer Schule. Für den späteren Wettbewerb wurden diese beiden Gruppen in obiger Konstellation angemeldet.\\


\section{Onlineveranstaltungen}
Aufgrund der COVID-19-Pandemie wurden einige Termine online abgehalten. Dafür trafen sich die Teilnehmer und die Autoren gemeinsam in einem Online-Raum, wobei öfters zwei Kinder sich zuhause getroffen haben und dann gemeinsam in der Sitzung dabei waren. Aufgrund von Ausfällen seitens der Autoren konnte nicht aus allen Terminen Beobachtungen gewonnen werden, da der Kursleiter nicht in der Lage war, den Kindern gleichzeitig etwas beizubringen und dabei auf alle Kinder zu achten.\\

Für die Online-Sitzung mit dem Thema Forschungssonde waren Lulu und Sara abwesend.\\
Heinz arbeitet unsorgfältig an seinen Bauwerken, was zur Folge hatte, dass sein Bauwerk öfters auseinander fiel. Nur mit Hilfe der Autoren gelang es ihm, am Ende ein korrektes Modell zu bauen. Der Grund für die Instabilität war in der Anleitung, die er nicht genau gelesen hatte. Sein Programm funktionierte. Moritz und Benny arbeiten ihr Modell schnell und mit viel Freude durch, am Ende existiert ein Modell, welches funktioniert. Durch die schnelle Arbeitszeit war geplant, ihnen weiterführende Aufgaben zu geben, was jedoch aufgrund ihrer plötzlichen Abwesenheit nach der Fertigstellung ihres Modells nicht mehr möglich war. Im Vergleich zu ihrem sonstigen Verhalten war Henriette während dieser Online-Sitzung deutlich effektiver und konzentrierter. Sie musste in dieser Sitzung ohne ihre sonstige Teampartnerin Lulu auskommen. Stattdessen hatte sie eine Freundin bei sich, welche sich bisher noch nie mit Lego WeDo auseinandergesetzt hatte. Gemeinsam bauten und programmierten sie das Modell, während Henriette ihrer Freundin alles erklärte. Die beiden Kinder Jonas und Mario kamen in dieser Sitzung nur sehr langsam voran. Jonas baute unter anderem sehr viele unnötige Anbauten an ihren Roboter an, bevor sie ein lauffähiges Programm hatten. Auch wenn sie am Ende ein fertiges Programm hatten, hatten sie beim Programmieren des Roboters sehr große Schwierigkeiten.\\
%-----------------
Bei einer weiteren Sitzung mit den Thema Tiermodelle arbeiteten Benny und Moritz, Mario und Jonas sowie Henriette und Lulu zusammen. Sara arbeitete alleine, Heinz war abwesend.\\
Sara arbeitete mit ihrer Mutter zusammen, welche am Ende des Kurses den Autoren berichtete, dass Sara die Programmierarbeit alleine übernommen hatte. In Partnerarbeit stellten Benny und Moritz ihr Modell sehr schnell fertig und auch die Programmierung funktionierte ohne Probleme. Die beiden Kinder sind jedoch danach direkt abwesend, so dass ihnen keine weitere Aufgabe gegeben werden konnte. Ebenfalls schnell fertig sind Mario und Jonas. Zusätzlich zur gestellten Aufgabe erweitern sie ihr Programm mit zusätzlichen Funktionen. Bei der Gruppe von Henriette und Lulu funktioniert die Zusammenarbeit nicht sehr gut. Henriette baut selbständig nicht relevante Bauwerke und hindert so Lulu daran, ihr Modell fertig zu bekommen. Beiden fehlt jedoch der Ideenreichtum, um ihr Modell erweitern zu können. Ihr Programm funktioniert auch erst nach etwas Hilfe durch die Autoren. Lulu musste dies alleine programmieren, da Henriette abgelenkt war.\\
%-----------------
Bei der Online-Sitzung mit dem Thema Tiermodelle \RNum{2} bauten die beiden Kinder Benny und Moritz ihr Modell zügig fertig. Dabei wurde beobachtet, dass die beiden mit sehr viel Spaß und Freude an die Sache herangingen. Für ihr Modell bauten sie zusätzliche Details mit ein, wie beispielsweise Geräusche, obwohl es nicht erfordert war. Nachdem sie ihr Modell beendeten, waren sie zwar noch im Raum angemeldet, jedoch erschienen die beiden nicht mehr vor der Kamera und waren auch nicht mehr erreichbar. In der Gruppe von Lulu und Henriette herrschte während der Arbeit eine schlechte Kommunikation. Henriette arbeitete nicht sehr konzentriert an ihrem Modell, sondern war damit beschäftigt, Unfug zu treiben. Mit den anderen Teilen baute Henriette andere Bauwerke, welche nichts mit den geforderten Aufgaben zu tun hatten. Henriettes Teampartnerin Lulu musste daher alles von alleine machen, weshalb sie nur sehr langsam voran kam. Das Programmieren funktionierte bei ihr sehr gut. Mario arbeitete sehr zügig an der Fertigstellung seines Modells, hatte auch mit der Programmierung des Modells keine Probleme. Bei Heinz war erkennbar, dass ihm die Fähigkeit der Fantasie fehlt. Er hatte große Schwierigkeiten damit, das Modell, das er baute, mit dem realen Abbild zu assoziieren. Er arbeitete in einem normalen Tempo, allerdings nicht sehr sorgfältig. Dies war unter anderem daran erkennbar, dass sein Modell sehr instabil war und ständig auseinander fiel.

\section{Verhalten am Wettbewerbstag}

Beim Wettbewerb der \acrshort{fll} wurden zwei Gruppen gebildet. Die erste Gruppe bestand aus den zwei Border Collies Mario und Jonas sowie dem Panda Sara. Heinz war ursprünglich ebenfalls in dieser Gruppe, konnte aber nicht an diesem Termin teilnehmen. Die Gruppe war sehr aufgeregt während ihrer Präsentation. Sara begann als Rednerin und hatte insgesamt auch den größten Redeanteil. Die beiden anderen Kinder der Gruppe waren etwas schüchterner und hielten sich deswegen zurück.\\
Ebenfalls aufgeregt war die Gruppe der beiden Erdmännchen, Benny und Henriette, und den beiden Elefanten Moritz und Lulu. In dieser Gruppe war der Redeanteil besser verteilt, auch wenn beispielsweise Moritz den wenigsten Anteil hatte. Dafür nannte er hin und wieder Details zu ihrer Arbeit. Die Präsentation dieser Gruppe begann Benny.

\section{Beobachtungen der German University of Cairo}
Im folgenden Abschnitt werden die Beobachtungen aufgeführt, welche im Rahmen einer Parallelveranstaltung mit ähnlichem Aufbau und Inhalt aufgezeichnet wurden. Diese stammen jedoch nicht von den Autoren selbst, sondern von Mitarbeitenden der \acrlong{guc}.\\

\subsection*{Abdelrahman}
Abdelrahman bevorzugte für den ersten Termin die Alleinarbeit. Im Verlaufe der weiteren Termine verbesserte sich dies, durch das Drängen der dortigen Kursleiter akzeptierte er die Arbeit mit seinen Partnern. Am Ende war die Wichtigkeit des Teams zwar noch nicht sehr gut, aber im Vergleich zum Beginn deutlich besser. Gleichzeitig lernte er zu Beginn nicht sehr viel im Team, da er erst seine Teampartner akzeptieren musste. Nach dem ersten Termin funktionierte dies deutlich besser, er fing an, die Ideen seiner Partner zu akzeptieren. Seine generelle Fähigkeit im Team zu arbeiten konnte nur durch Drängen verbessert werden, er selbst arbeitete von sich aus nicht gerne im Team. Gegen Ende ist dies deutlich besser. Abdelrahman gab generell gutes Feedback und fragte auch sehr aktiv nach Rückmeldungen. Jedoch war es für ihn schwierig, andere in sein Team mit einzubinden. Die Ideen seiner Partner wurden gegenüber seinen eigenen Ideen wenig geschätzt, allerdings wurde dies mit der Zeit besser. Ebenso bereitete es Abdelrahman zu Beginn Probleme, die andern Mitglieder seiner Gruppe als gleichwertig anzusehen. Über den Zeitraum, in dem die Kurse stattfanden, konnte er in diesem Aspekt Fortschritte aufzeigen. Seine Kommunikation mit anderen konnte, trotz Schwierigkeiten in der Zusammenarbeit, als gut bezeichnet werden. Auch wenn Abdelrahman in der Lage war, in seiner Gruppe als Führungsperson voranzugehen, bevorzugte er es, dass dies andere Kinder übernahmen.\\

Abdelrahman war in der Lage, viele neue Ideen während den Veranstaltungen zu generieren. Er versuchte stets, dass seine Ideen sich von den Ideen seiner Partner unterschieden. Wenn man Abdelrahman dazu drängte, Ideen aus einer anderen Sichtweise zu generieren, schaffte er dies, selbständig war dies jedoch nicht möglich.\\
Für ihn war es kein Problem, zu abstrahieren. Seine Beharrlichkeit muss hierbei als besonders gut hervorgehoben werden, da die Beobachter für den ersten Termin für diese Kategorie ein "`very very very persistent"' notierten. Zu Beginn beobachteten die Kursleiter ein Verbesserungspotenzial für Abdelrahmans Integrationsfähigkeit, ab dem zweiten Termin veränderte sich dies und seine Fähigkeit wurde als gut eingestuft.\\
Abdelrahman ist zu Beginn der Kurse nicht sehr offen gegenüber Neuem, was sich aber mit der Zeit verbessert und er Neues akzeptiert. Seine emotionale Sensitivität konnte nur schlecht beobachtet werden, genauso wie Humor und Fantasie.\\
Seine Logikfähigkeiten sind vorhanden, zu Beginn denkt er ein wenig zu logisch, in den restlichen Terminen verbessert sich dies und er zeigt gutes bis sehr gutes logisches Denken.\\ 

Für die Dekomposition konnte Abdelrahman selbständig das Problem beschreiben und versteht, was das Erfolgskriterium ist. In weiteren Terminen ist er sogar in der Lage, zu den Erfolgskriterien eigene Ideen hinzuzfügen. Zudem fügt er beim Zerlegen des Problems seine eigenen Ideen hinzu.\\
Bei der Generalisierung versteht er meistens die Muster und verbessert seine Fähigkeit, gelernte Konzepte nicht nur anzuwenden, sondern auch durch eigene Ideen zu verbessern.\\
Abdelrahman zeigte während den Terminen, dass er in der Lage war, eine Abfolge größerer Schritte zu definieren und einzelne Schritte zu implementieren. Algorithmische Konzepte verstand er aber erst während des letzten Termins.\\
Seine Programme funktionierten immer ohne Hilfe, Probleme wurden selbständig von ihm gelöst. Zudem konnte er alternative Lösungen präsentieren und erklären, warum seine Lösung das Ausgangsproblem löste.\\
Über die ganzen Termine hinweg war Abdelrahman in der Lage, das wichtigste Segment der Lösung sowie das wichtigste Detail zu nennen. Zudem konnte er ohne Hilfe erklären, wie die Lösung mit dem vorher definierten Erfolgskriterium zusammenhängt.\\

\subsection*{Farid}
Für Farid war das Arbeiten mit seinem Teampartner sehr wichtig. Jedoch war für ihn sehr wichtig, dass sein Teampartner eine bestimmte Person ist, ansonsten war er gegenüber Teamarbeit nicht sehr offen. Dies verbesserte sich im Laufe der Zeit, so dass er andere Partner akzeptierte. Generell war er jedoch in der Lage, in der Gruppe mehr zu lernen als alleine und war, trotz der Skepsis gegen anderen Teampartnern, fähig, im Team zu arbeiten. Es wurde beobachtet, dass er gutes Feedback gab und auch konstant nach Rückmeldungen von anderen Personen fragte. Farid versuchte, andere Personen in seine Gruppe mit einzubinden, für die Beobachter war es auch wichtig zu notieren, dass er die anderen Gruppenmitglieder respektierte. Der Einschätzung der Beobachter nach ist Farid ein sehr guter Kommunikator. Seine Rolle in Gruppen ist weniger die Führungsrolle, sondern eher die Rolle, bei der er die Anweisungen der Führungsperson durchführt. Führen selbst kann Farid nicht sehr gut.\\

Wenn es um das Generieren von Ideen geht, ist Farid in der Lage, sehr viele Ideen zu generieren. Diese sind allerdings nicht immer strukturiert und der Gedankengang ist nicht immer nachvollziehbar. Seine Ideen sind zwar häufig Nachahmungen anderer Ideen, jedoch versucht Farid, die Ideen mit seinen eigenen Ideen abzuändern. Generell ist er auch sehr flexibel, was Ideen angeht, auch wenn er dafür etwas Hilfe benötigt.\\
Farid zeigte während den Terminen ein sehr detailliertes Arbeiten, auch wenn das Erkennen des höheren Zieles dadurch etwas schlechter wurde. Deshalb musste er dafür etwas mehr geführt werden. Auch Farid arbeitete sehr beharrlich an den Projekten. Die Fähigkeit zu integrieren konnte bei ihm beobachtet werden.\\
Farid zeigte nach Überzeugungsarbeit sich offen gegenüber Neuem, gegen Ende öffnete er sich von selbst ein wenig mehr den Ideen und Vorschlägen anderer. Bei Farid zeigte sich ein guter Sinn für Humor. \\
Beim ersten Termin wurde bei Farid ein sehr gutes logisches Verhalten dokumentiert, der Auffassung der Beobachtenden nach müsste Farid in weiteren Terminen mehr gefordert werden. Im weiteren Verlauf wurde dies nicht mehr benötigt.\\

Bei den Computational Thinking Skills zeigte Farid, dass er in der Lage war, ein Problem zu beschreiben, die Erfolgskriterien zu benennen und das Problem in kleinere Schritte zu zerlegen. Bei letzterem war auch er in der Lage, dies durch weitere, eigene Ideen zu ergänzen.\\
Muster erwähnte Farid nie, benutzte jedoch selbständig das Wissen über gelernte Konzepte und wandte dies an.\\
Ebenso selbständig konnte Farid große Schritte für die Lösung eines Problems definieren und auch kleine Schritte implementieren. Das Verständnis für algorithmischen Konzepte konnte erst am letzten Termin beobachtet werden, davor war Farid dazu nicht fähig.\\
Die Programme, die Farid erstellte, funktionierten korrekt. Probleme wurden selbständig behoben. Durch eigene Ideen ergänzte Farid alternative Lösungen und die Beschreibung, wie seine Lösung das Problem löst.\\
Die wichtigsten Details und Bauteile der Lösung konnte Farid ohne Hilfe selbständig erklären. Für ihn war es ebenfalls kein Problem, eine Verbindung zwischen der Lösung und dem Erfolgskriterium zu ziehen. 

\subsection*{Gamal}
Das Arbeiten im Team war für Gamal wichtig. Er lernte sehr gut im Team und zeigte auch, dass er durchaus fähig war, im Team zu arbeiten. Aber für ihn war es sehr wichtig, mit einem bestimmten Partner zu arbeiten, er weigerte sich, mit anderen zusammenzuarbeiten. Wenn er jedoch im Team arbeitete, respektierte er die Rollen seiner Partner. Von sich aus fragte Gamal zu Beginn nicht nach Feedback, nur nach einiger Nachfrage fragte er nach. Dies verbesserte sich, er fragte in den weiteren Terminen nach Feedback am Ende der Session. Er selbst gab jedoch gutes Feedback und konnte den Fortschritt oder aktuelle Probleme beschreiben. Seine Anstrengung, andere mit einzubinden, waren in einem guten Rahmen. Gegenüber den Ideen anderer war er skeptisch und musste erst überzeugt werden. Später verbesserte sich dies und er fragte aktiv nach den Beiträgen anderer. Alle Mitglieder wurden von ihm gleich behandelt. Ebenso verbesserte sich seine Kommunikation von einem Austausch auf Nachfrage bis hin zu einer guten Kommunikation. In den gesamten Terminen erwies sich Gamal als ein guter Anführer. \\
Gamal generiert im Verlauf der Kurse an der \acrshort{guc} sehr viele gute Ideen, welche unterschiedlich sind von den Ideen der anderen Kinder. Gegenüber der Replikation von Ideen ist er abgeneigt. Im Zeitraum der Kurse entwickelt sich seine Flexibilität, so dass er am Ende selbständig in der Lage ist, andere Perspektiven zu haben. \\
Das Kind ist sehr detailorientiert und verliert so den Überblick über das große Ganze. Im Verlaufe der Kurse verbessert sich aber diese Fähigkeit. Zudem ist er sehr beharrlich in den Projekten. Seine Integrationsfähigkeit wurde von den Beobachtern als gut beschrieben, welche konstant bleibt.\\
Gamal ist offen gegenüber Neuem und wird über den Zeitraum der Kurse offener im Bereich des Humors, so dass er am Ende als sehr humorvoll beschrieben wird.\\
Seine Logikfähigkeit bleibt während den Kursen konstant auf einem guten Niveau.\\
In fast allen Bereichen des Computational Thinkings fügt Gamal eigene Ideen hinzu, in allen andern Fällen ist er zumindest selbständig in der Lage, diese Attribute anzuwenden. Lediglich im Bereich des algorithmischen Denkens benötigt er fast bis zum Ende, um die Konzepte zu verstehen.

\subsection*{Omar}
Omar wird von den Beobachtern im Bereich des Teamworks in den meisten Fällen nur bescheinigt, dass er diese Teilbereiche zeigt, jedoch fügen sie nur wenig eigene Notizen hinzu. Lediglich sein Feedback wird zu Beginn kritisiert, was sich jedoch im Verlauf der Kurse ändert. Auch die Kommunikation zwischen Omar und den anderen Kindern verbessert sich wohl im Laufe der Zeit. Er wird jedoch von den Beobachtern an der \acrshort{guc} nicht als Führungsperson gesehen.\\
Omar entwickelt im Verlaufe der Termine eine immer höhere Anzahl an generierten Ideen. Seine Ideen sind jedoch häufig Replikate anderer Ideen, jedoch versucht er, sich zu verbessern, so dass am Ende die Verteilung von eigenen Ideen und replizierten Ideen sich verbessert. Seine Fähigkeit der Flexibilität wird jedoch als sehr gut eingeschätzt, nachdem er zu Beginn Probleme hatte.\\
Für Omar war es leichter, das Abstrakte zu erkennen und nicht die spezifischen Details. Gegen Ende wird er etwas detailorientierter, jedoch ohne die Abstraktionsfähigkeit zu verlieren. Seine Beharrlichkeit wird als durchschnittlich bezeichnet, seine Integrationsfähigkeit sehen die Beobachter auf einem guten Niveau.\\
Omar ist ein sehr offenes Kind gegenüber neuen, anderen Ideen und Ansätzen. Sein Sinn für Humor wird den Beobachtungsbögen der ägyptischen Veranstaltung nach als gut definiert.\\
Wenn es um die logische Denkfähigkeit geht, ist Omar weniger ein logisch denkendes sondern ein kreativ denkendes Kind, was sich im Verlaufe der Kurse nicht verändert.\\
Die Computational Thinking Skills von Omar sind in der Regel so ausgeprägt, dass er bei fast allen Aspekten in der Lage ist, diese selbständig anzuwenden, ohne dass Hilfe benötigt wurde. Größere Probleme bereiteten ihm jedoch das Zerlegen von Problemen, die Implementierung von Schritten und das Anwenden von Konzepten. Bis auf den letzten Aspekt verbessert sich Omar in diesen Bereichen im Verlaufe der Veranstaltung. Beim Aspekt der alternativen Lösung ist Omar am Ende in der Lage, Ideen hinzuzufügen statt nur selbständig Lösungen zu nennen.