\chapter{Grundlagen}

In den folgenden Abschnitten werden Grundlagen gelegt, welche für das weitere Verständnis der Studienarbeit benötigt werden.

\section{LEGO}
\gls{Lego}
Hier steht nachher Zeugs, was mit Lego zu tun hat. Dazu gehört die WeDo-Kästen und auf jeden Fall der Wettbewerb, zumindest sollte definiert werden, warum die Kinder eigentlich mitmachen.

\section{Persönlichkeitscharaktere}

Der folgende Abschnitt geht genauer auf den Test für den Persönlichkeitstypus der einzelnen Kinder ein.

Der Test, den die Kinder durchgeführt haben, ordnet jedem Teilnehmer eine Tier zu. Jedes dieser Tiere steht für einen Typus nach dem \acrlong{mbti} (\acrshort{mbti}). Der \acrshort{mbti} besteht aus 16 verschiedenen Persönlichkeitstypen (vgl. Abbildung \ref{img:mbti}), bei denen jede der Typ die Überlagerung aus vier verschiedenen Attributen ist. Wie in der Abbildung dargestellt, gibt es damit 16 verschiedene Anordnungen der Zeichen E, F, I, J, N, P, S und T, wobei jedes Zeichen für ein bestimmtes Attribut steht. Insgesamt werden aus diesen Zeichen vier Paare gebildet: EI, SN, TF und JP. In einer Abkürzung nach Myer-Briggs kann jeweils nur ein Buchstabe eines Paares vorkommen, woraus die 16 Typen in Abbildung \ref{img:mbti} resultieren.
\begin{figure}[htbp!]
	\centering
	\fbox{\includegraphics[width=0.35\textheight,angle=0]{img/Myers-Briggs-Typenindikator}}
	\caption[Myers-Briggs-Typenindikator]{Die 16 Persönlichkeitstypen nach Myers-Briggs}
	\label{img:mbti}
\end{figure}

Jeder einzelne Typ hat nach Myers-Briggs Auswirkungen auf Personen. \cite{myers_myers_2002} 
\begin{figure}[htbp!]
	\centering
	\fbox{\includegraphics[width=0.35\textheight,angle=0]{img/distribution}}
	\caption[Verteilung der Myers-Briggs-Typen]{Verteilung der Myers-Briggs-Typen \cite{myers_myers_2002}}
	\label{img:mbti_distribution}
\end{figure}

\subsection{Studienrelevante Persönlichkeitscharaktere}
\subsubsection{Border Collie - ESTJ / ENTJ}
\begin{figure}[htbp!]
	\centering
	\fbox{\includegraphics[width=0.2\textheight,angle=0]{img/Border_Collie}}
	\caption[Border Collie]{Border Collie}
	\label{img:Border_Collie}
\end{figure}
„Collie“ ist das schottisch-gälische Wort für „nützlich“. Sie sind logisch, entschlossen, kompetent und organisiert und bevorzugen eine Umgebung, in der sie ihre Intelligenz einsetzen, andere führen und aktiv gehalten werden können. Diese Kinder sind am glücklichsten, wenn sie die Möglichkeit haben, die Verantwortung zu übernehmen und sich zu messen. Sie genießen Herausforderungen, Debatten und die Interaktion mit einer Vielzahl von Menschen. Das Leben ist ein großer Wettbewerb, und sie sind entschlossen zu gewinnen.

\subsubsection{Elefant - ESFJ / ENFJ}
\begin{figure}[htbp!]
	\centering
	\fbox{\includegraphics[width=0.2\textheight,angle=0]{img/Elephant}}
	\caption[Elefant]{Elefant \cite{knowAndLove}}
	\label{img:Elefant}
\end{figure}
Freundlich, aufgeschlossen und organisiert. Sie sind am glücklichsten, wenn sie anderen helfen, gesellschaftliche Veranstaltungen planen oder an Aktivitäten teilnehmen, an denen Menschen beteiligt sind – ALLE Menschen. Diese Kiddos bevorzugen ein harmonisches und kooperatives Umfeld mit viel Lob und Zuneigung. Im Leben dreht sich alles um echte Beziehungen und die Verbindung zu Menschen. 

Friendly, outgoing and organized. They're happiest when helping others, planning social events or participating in activities that involve people - ALL of the people. These kiddos prefer a harmonious and cooperative environment with lots of praise and affection. Life is all about genuine relationships and connecting with people.	\\
\subsubsection{Erdmännchen - INFP / ISFP}
\begin{figure}[htbp!]
	\centering
	\fbox{\includegraphics[width=0.2\textheight,angle=0]{img/Meerkat}}
	\caption[Erdmännchen]{Erdmännchen \cite{knowAndLove}}
	\label{img:Elephant}
\end{figure}
Moralisch, sanft und sensibel mit einer kreativen und doch komplexen inneren Welt. Sie sind am glücklichsten in einer ruhigen, kooperativen und unterstützenden Umgebung, in der sie sinnvollen Dingen nachgehen können. Diese Kinder brauchen viel eingeplante Zeit allein, um ihre natürlichen Bauchgefühle zu verarbeiten und die Welt zu analysieren. Sie sind tief im Einklang mit den Gefühlen und Bedürfnissen anderer und neigen dazu, eine Friedensstifterrolle einzunehmen. Im Leben geht es darum zu verstehen, wie Menschen ticken. 

Moralistic, gentle and sensitive with a creative yet complex inner world. They're happiest in calm, cooperative and supportive environments where they can pursue meaningful matters. These kiddos need plenty of scheduled alone time to process their natural gut feelings and analyze the world. They are deeply in tune with others' feelings and needs and tend to take on a peacemaker role. Life is about understanding what makes people tick. \\	
\subsubsection{Panda - INFJ / INTJ}
\begin{figure}[htbp!]
	\centering
	\fbox{\includegraphics[width=0.2\textheight,angle=0]{img/Panda}}
	\caption[Panda]{Panda \cite{knowAndLove}}
	\label{img:Panda}
\end{figure}
Intensiv privat, kreativ und ideenorientiert. Sie sind am glücklichsten, wenn ihre Bemühungen und einzigartigen Ideen anderen helfen, zu wachsen und zu lernen. Sie schätzen Autonomie und sind vertrauenswürdige, einzigartige, fähige und aufschlussreiche Personen. Diese Kinder bevorzugen eine ruhige Umgebung, in der sie in ihrer inneren Welt denken, verarbeiten und erschaffen können. Den Sinn hinter den Dingen zu verstehen, ist ihr Lebensziel. 

	Intensely private, creative and idea-oriented. They're happiest when their efforts and unique ideas help others to grow and learn. They appreciate autonomy and being trustworthy, unique, capable and insightful individuals. These kiddos prefer a calm environment where they can think, process and create in their inner world. Understanding the meaning behind things is their goal in life. \\
\subsection{Restliche Tiere}
Im folgenden wird zur Einordnung der bereits genannten Tiere in das gesamte Spektrum der Persönlichkeitstypen mit den Tieren, welchem keinem Kind in der Studie zugeordnet wurde, dargestellt.



\section{Computational Thinking} \label{sec:ct}
Um das informatische Denken der Kinder messen zu können, wurde mit den Kindern der Computational Thinking Test durchgeführt. Da die Kinder noch sehr jung sind, wurde deshalb die spezielle Variante für Jüngere, der sogenannte Beginners Computational Thinking Test, von den Kindern absolviert. Nach den Autoren des \acrshort{bctt}, \citeauthor{bcct}, ist die Fähigkeit des Computational Thinkings, eine kognitive Fähigkeit, welche für eine erfolgreiche Anpassung an die Zukunft wichtig ist, eine Kernfähigkeit. Dadurch, dass die Welt immer mehr den Schwerpunkt auf Technologie und Programmierungen legt, ist diese Fähigkeit des Computational Thinkings eine sehr kritische. Die Autoren des \acrshort{bctt} halten es für wichtig, dass Schüler und Schülerinnen in der Lage sein sollten, kritisch zu denken und komplexere Probleme lösen zu könne.\\
Die vier Kernelemente des Computational Thinkings sind Abstraktion, Dekomposition, Algorithmen und das Debugging.\\
\cite{bcct} 
\subsection{Beginners Computational Thinking Test}
Der \acrlong{bctt}, kurz \acrshort{bctt}, ist eine Weiterentwicklung des Computational Thinking Tests für Jüngere. Gemeinsam mit Experten wurde eine erste Version erstellt, welche mithilfe von Pilottests und Experten in eine zweite Variante verbessert wurde. Diese wurde anschließend an verschiedenen Schulen in Spanien durchgeführt.\\

Der \acrshort{bctt} ist in mehrere Segmente aufgeteilt. Insgesamt sind es drei große Kerngebiete, die wiederum in sechs verschiedene Aufgabengebiete aufgeteilt wurden, die der Test überprüft. Das erste Segment \glqq Sequences"' überprüft, ob die Kinder in der Lage sind, Sequenzen zu folgen. Dies ist das erste Kerngebiet des \acrshort{bctt}. Der zweite Abschnitt überprüft einfache Schleifen, der dritte Abschnitt überprüft verschachtelte Schleifen. Die beiden genannten Abschnitte bilden damit die Überprüfung für Schleifen. Das letzte Kerngebiet sind die sogenannten Conditionals. Dieses Kerngebiet wird von den drei Gebieten If-Then, If-Then-Else und While gebildet. Dort wird überprüft, ob die Kinder in der Lage sind, Anweisungen, welche an Bedingungen geknüpft sind, korrekt auszuführen.\\

Die Gebiete sind nicht gleich groß. Die Größe der Aufgabengebiete ist in der untenstehenden Tabelle dargestellt. Der größte Bereich wird vom Kerngebiet Schleifen gebildet, gefolgt von den Conditionals. Das kleinste Kerngebiet sind die einfachen Sequenzen.

\begin{table}[H]
	\centering
	\begin{tabular}{|l|l|}
		\hline
		Bezeichnung              & Anzahl Fragen \\ \hline
		Sequenzen                & 6             \\ \hline
		einfache Schleifen       & 5             \\ \hline
		verschachtelte Schleifen & 7             \\ \hline
		If-Then                  & 2             \\ \hline
		If-Then-Else             & 2             \\ \hline
		While                    & 3             \\ \hline
	\end{tabular}
	\caption{Anzahl Fragen pro Aufgabengebiet}
	\label{tab:distQuestions}
\end{table}

In Abbildung \ref{img:bcttQuestion} ist beispielhaft eine Frage eines Aufgabengebiets, hier das Gebiet der While-Conditionals, dargestellt. Eine Frage besteht immer aus einer Grafik und vier dazugehörigen Antworten, von denen nur eine richtig ist. Bei den Antwortmöglichkeiten sind die jeweiligen Anweisungen dargestellt, welche zur Lösung führen sollen. Die Aufgaben, bei denen neben Pfeilen auch andere Symbole verwendet werden, wie in der untenstehenden Abbildung dargestellt, beinhalten noch eine zusätzliche Anweisung für die Person, die den Test bearbeitet. 

\begin{figure}[H]
	\centering
	\fbox{\includegraphics[width=0.4\textheight,angle=0]{img/bcttQuestion}}
	\caption[Beispielhafte Frage des \acrshort{bctt}]{Beispielhafte Frage des \acrshort{bctt}\cite{bcct}}
	\label{img:bcttQuestion}
\end{figure}

Die zweite Version des Tests wurde so abgeändert, dass auch Farbenblinde diesen Test durchführen können, weshalb die Felder in Abbildung \ref{img:bcttQuestion} mit Symbolen versehen wurden statt wie ursprünglich nur farbig dargestellt. Dies hat den Vorteil, dass zusätzlich der Test nicht zwingend in Farbe gedruckt werden muss, sondern auch in schwarz-weiß gut lösbar ist. \cite{bcct}

Der Test wurde wie bereits erwähnt an mehreren Schulen in Spanien angewendet. Insgesamt wurde eine Stichprobe von 299 Schüler und Schülerinnen an drei verschiedenen Schulen durchgeführt. Die Schüler und Schülerinnen waren in verschiedenen Klassenstufen, so dass der Test die Klassen 1,2,4,5 und 6 abdeckte. Für die Studienarbeit sind die Daten der Schüler und Schülerinnen relevant, welche die zweite Variante des \acrshort{bctt} durchführten. Die Ergebnisse nach \citeauthor{bcct} sind in der untenstehenden Tabelle dargestellt. Für die Studienarbeit nicht relevante Daten wurden aus Gründen der Übersichtlichkeit nicht übertragen.

\begin{table}[H]
	\centering
	\begin{tabular}{|l|l|l|l|}
		\hline
		\rowcolor[HTML]{C0C0C0} 
		Klassenstufe & N  & durchschnittliches Ergebnis & Standardabweichung \\ \hline
		2            & 18 & 14.278                      & 2.445              \\ \hline
		4            & 28 & 21.286                      & 1.922              \\ \hline
		6            & 25 & 21.280                      & 3.542              \\ \hline
	\end{tabular}
	\caption[Statistiken des \acrshort{bctt}]{Statistiken des \acrshort{bctt} nach \citeauthor{bcct}}
	\label{tab:statisticsBCTT}
\end{table}


\section{Torrance Tests}{
	\label{sec:torrance_tests}
	\begin{figure}[H]
		\centering
		\fbox{\includegraphics[width=0.4\textheight,angle=0]{img/ttct}}
		\caption[Relation von TTCT-Maßnahmen]{Relation von TCTT-Maßnahmen gegenüber kreativem Denken und kreativer Haltung \cite{Kim2016}}
		\label{img:ttct}
	\end{figure}
}
