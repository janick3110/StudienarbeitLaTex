\chapter{Intro}


\section{Motivation}

In dem Artikel \citetitle{stephanidis_seven_2019} werden die großen Herausforderungen aufgeführt, welche an Maschinen gestellt werden, mit denen der Mensch interagiert. Immer schneller schreitet der technologische Fortschritt voran. Dies führt nicht nur dazu, dass die Erwachsenen sich auf die neue Technologie stetig neu einstellen müssen, sondern auch dazu, dass die Kinder von heute vor immer komplexere Probleme gestellt werden, die diese Technologie mit sich bringt. Und so führen \citeauthor{stephanidis_seven_2019} \textit{Lernen und Kreativität} als eine dieser Großen Herausforderungen an. Besonders die Informatik ist von diesem Problem betroffen. Schüler entwickeln oft nur oberflächliche Kenntnisse, anstelle von Strategien zum Lösen von Problemen \cite{kazimoglu_serious_2012}.\\
Doch die Technologie stellt nicht nur Herausforderungen, sondern bietet auch Chancen. So lassen sich Technologien in den Lernprozess integrieren, dieser lässt sich sogar durch die Technologie leiten. Dies ist nichts Neues. Bereits im Jahr 2000 gab es mit der \textit{Addy}-Reihe zahlreiche Lernspiele für verschiedene Schulfächer und Altersstufen \cite{addy} und auch jüngere Altersgruppen wurden bereits an die Informatik herangeführt, indem Bausteine so mit Elektronik versehen wurden, dass je nach dem wie die Steine aufeinander gesteckt werden, unterschiedliche Aktionen ausgeführt werden \cite{wyeth_tangible_2002}. \\
Doch auch in der jüngeren Vergangenheit gibt es neue Ansätze und Methoden, um Kinder aller Altersstufen an verschiedene Lernthemen heranzuführen, wie ein Programm zum Üben der Rechtschreibung an Grundschulen \cite{berkling_learning_2020}.\\
Diesen Trend hat auch \gls{Lego} erkannt, die mit \gls{Lego}-Education seit dem Jahr 1999 eine eigene Produktsparte nur für den Zweck, an Schulen eingesetzt zu werden, anbieten. Die Effektivität dieser Produkte wird von zahlreichen wissenschaftlichen Arbeiten erörtert und belegt \cite{perez_new_2015, karatrantou_algorithm_2008, jun_design_2016, cuellar_design_2014, klassner_lego_2003}\\
Ein weiterer Aspekt des Lernens mithilfe von Technologie ist die Möglichkeit, dass sich die Lernumgebung an den Lernenden anpasst. Beliebt ist hierbei der \acrlong{mbti}. \citeauthor{yel_adaptive_2018} haben beispielsweise ein Modell erstellt, bei dem sich die genannten Lernabschnitte(\textit{Einführung}, \textit{Inhalt}, \textit{Übertragung} und \textit{Übung}), nach dem \acrshort{mbti} an den Lernenden anpassen \cite{yel_adaptive_2018}.\\
In dieser Arbeit soll qualitativ der Fortschritt einer Gruppe von Kindern dokumentiert und untersucht werden, während der Vorbereitung auf den Wettbewerb \acrlong{fll}. Dabei soll insbesondere auf Zusammenhänge zwischen der Entwicklung in den verschiedenen Kategorien (Teamwork, Computational Thinking) und dem \acrshort{mbti} der Kinder geachtet werden.



\section{Beschreibung des Projektablaufes}
Im Rahmen des Projektes wurden im wöchentlichen Abstand regelmäßig Kurse mit Lego WeDo 2.0. durchgeführt. Durch die Autoren der Studienarbeit wurde die Mitarbeit der Kinder mithilfe von im Vorhinein erstellten Beobachtungsbögen beobachtet. Diese legen einen Fokus unter anderem auf die Motivation, Kreativität sowie auf die Teamarbeit und das algorithmische Denken. Die Kinder führten dazu im Vorhinein einen Test durch, welcher sie in unterschiedliche Persönlichkeitskategorien unterteilt. Ebenfalls wurde zu Beginn der Test von \citeauthor{zapata_bctt_2021} in \citetitle{zapata_bctt_2021} durchgeführt, um das algorithmische Denken der Kinder zu untersuchen.
Um die Ergebnisse nicht mit den Kindern in Verbindung zu bringen, bekam jedes Kind ein Pseudonym. Die Pseudonyme wurden zum Teil von den Eltern der Kinder bei der Anmeldung festgelegt, die restlichen Namen wurden durch die Autoren aufgefüllt.\\
Gleichzeitig fand eine Zusammenarbeit der Autoren mit einer Parallelveranstaltung an der \acrlong{guc}(\acrshort{guc}) statt, die bei ihrer Vorbereitung auf die \acrshort{fll} ebenfalls den Fortschritt der Kinder in den Beobachtungsbögen festhielten.


\section{Daten}
Die Rohdaten, die für die Studienarbeit verwendet wurden, werden nach Abschluss der Arbeit auf GitHub hochgeladen, so dass diese verfügbar sind für eventuelle weitere Arbeiten, die auf dieser basieren. Der Link dazu ist folgender: \url{https://github.com/janick3110/technologyenhancedLearningData}