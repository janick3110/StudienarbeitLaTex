\chapter{Intro}


\section{Motivation}

In dem Aktikel \citetitle{stephanidis_seven_2019} werden die großen Probleme aufgeführt, welche sich in unserer modernen Welt ergeben. Immer schneller schreitet der technologische Fortschritt voran.



\section{Beschreibung des Projektablaufes}
Im Rahmen des Projektes wurden im wöchentlichen Abstand regelmäßig Kurse mit Lego WeDo 2.0. durchgeführt. Durch die Autoren der Studienarbeit wurde die Mitarbeit der Kinder mithilfe von im Vorhinein erstellten Beobachtungsbögen beobachtet. Diese legten einen Fokus unter anderem auf die Motivation, Kreativität und auf die Teamarbeit. Die Kinder führten dazu im Vorhinein einen Test durch, welcher sie in unterschiedliche Persönlichkeitskategorien unterteilt. Um die Ergebnisse nicht mit den Kindern in Verbindung zu bringen, bekam jedes Kind ein Pseudonym. Die Pseudonyme wurden zum Teil von den Eltern der Kinder bei der Anmeldung festgelegt, die restlichen Namen wurden durch die Autoren aufgefüllt.\\
Gleichzeitig fand eine Zusammenarbeit der Autoren mit einer Parallelveranstaltung an der \acrlong{guc}(\acrshort{guc}) statt.\\


\section{Daten}
Die Rohdaten, die für die Studienarbeit verwendet wurden, werden nach Abschluss der Arbeit auf GitHub hochgeladen, so dass diese verfügbar sind für eventuelle weitere Arbeiten, die auf dieser basieren. Der Link dazu ist folgender: \url{https://github.com/janick3110/technologyenhancedLearningData}