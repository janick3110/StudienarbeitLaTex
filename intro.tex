\chapter{Intro}


\section{Zieldefinition}
\cite{SevenHCIGrandChallenges}




\section{Beschreibung des Projektablaufes}
\gls{report} hallo Test
Im Rahmen des Projektes wurden im wöchentlichen Abstand regelmäßig Kurse mit Lego WeDo 2.0. durchgeführt. Durch die Autoren der Studienarbeit wurde die Mitarbeit der Kinder mithilfe von im Vorhinein erstellten Beobachtungsbögen beobachtet. Diese legten einen Fokus unter anderem auf die Motivation, Kreativität und auf die Teamarbeit. Die Kinder führten dazu im Vorhinein einen Test durch, welcher sie in unterschiedliche Persönlichkeitskategorien unterteilt. Um die Ergebnisse nicht mit den Kindern in Verbindung zu bringen, bekam jedes Kind ein Pseudonym. Die Pseudonyme wurden zum Teil von den Eltern der Kinder bei der Anmeldung festgelegt, die restlichen Namen wurden durch die Autoren aufgefüllt.\\
Sollte die Möglichkeit bestehen, findet eine Zusammenarbeit der Autoren mit einer Parallelveranstaltung an der \acrlong{guc}(\acrshort{guc}) statt.\\
Um eine größere Teilnehmerzahl erreichen zu können, soll auf den Wettbewerben der First Lego League mithilfe von Umfrageformularen Teilnehmer anderer Gruppen befragt werden, so dass diese Ergebnisse mit der an der Dualen Hochschule Karlsruhe durchgeführten Studie verglichen werden können..
